\section{Background}


Migrating the telephone system away from a very centralized Public Switched Telephone Network (PSTN) system to a much more distributed Voice over IP (VoIP) system has a lot of benefits. These come from allowing developers the freedom of innovation and creativity to develop their own solutions. However, precisely because of this added freedom security becomes a problem. Malformed code and misconfiguration can become a huge issue and thus open up new security vulnerabilities \cite{voipbg}. It is worth noting that a staggering 88\% of the security vulnerabilities arise from implementation details of VoIP \cite{keromytis}.

In the traditional PSTN model the security vulnerabilities are strongly limited by physical resource access, but in the case of VoIP systems the weakest link can be present on any layer of the system: the transport protocols, the VoIP devices, the VoIP application, or even the operating system. The user this has to rely on additional security measures such an encryption \cite{voipbg}. 

Additionally, attack techniques such as Denial of Service (DoS) suddenly become more relevant in a VoIP system because of the structure of the Internet Protocol. Places which allow cross communication between VoIP and PSTN systems will affect both the systems in ways not thought of before \cite{voipbg}.

In a survey done on VoIP vulnerabilities \cite{keromytis}, the three major security issues were : DoS attacks, man-in-the-middle-attack, social threats. However, the proportion of research papers to the importance of the issue (by occurrences) was not equal. For example, the DoS attack accounted for 58\% of the issues, but only 21\% of papers addressed the problem. Conversely, social threats such as Spam over Internet Telephony (SPIT) accounted for 18\% of the issues, but only 50\% of papers focused on this problem. 

