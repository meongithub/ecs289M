\section{Related Work}

Because no central authority controls the way VoIP is implemented, vulnerabilities related to confidentiality, integrity, and availability (CIA) arise in potentially all protocol layers ranging from physical, internet, transport to application. \cite{mcgann}. 

SIP (Session Initiation Protocol) is the application layer protocol used in VoIP here at UC Davis for signaling calls. The SIP specification is devoid of any security mechanism, and instead suggests importing well-known method \cite{geneiatakis2006survey}. Thus, in the absence of such a mechanism, there are possibilities of attacks such as spoofing, session hijacking, privacy, and several others.

Even with security mechanisms in place, SIP based VoIP is vulnerable to DoS attacks. The flooding attack technique can be used to the various terminals in a VoIP network which includes the registrar server, the proxy server or the end-user terminal \cite{geneiatakis2006survey}. SIP follows a 3-way handshake, so attackers can create spoofed IP addresses and request connections to the SIP server \cite{srihari}. Since it is an invalid request, the search from the sever will fail. If an attacker repeats this procedure a lot within a small time frame, the servers will be drained up. While this research is interesting for the operators of the UC Davis network, our focus relies more on attacking the confidentiality and integrity of the network rather than the availability. Shihari et al. describe also methods for registration hijacking \cite{srihari}. In our case, all clients get their certificates they use for registration offline. Therefore, this group of techniques will not work for us.

Vuong and Bai describe possible attacks on alternative protocols H.323, SIGTRAN and Megaco \cite{vuong}. These attacks include spoofing, DoD and snooping, similar to the one described above, but are specific to the protocols. Since these are not used within the UC Davis campus network, we direct the reader to the original source for further details.

Keromytis explains several attacks against encrypted VoIP streams \cite{keromytis2012comprehensive}. These include flow analysis, in which either certain bytes are added to the stream to follow it or certain parts are artificially delayed by a few milliseconds to de-anonymize communication partners. Another proposal is to use machine learning techniques to predict language from encrypted streams. Since these attacks are more advanced, they will become more interesting for us in a later stage of our project. Our focus lies on man-the-middle and privacy attacks on SIP.